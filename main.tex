%\title{Modelo de Projeto de pesquisa}
%% abtex2-modelo-projeto-pesquisa.tex, v-1.9 laurocesar
%% Copyright 2012-2013 by abnTeX2 group at http://abntex2.googlecode.com/ 
%%
%% This work may be distributed and/or modified under the
%% conditions of taTeX Project Public License, either version 1.3
%% of this license or (at your option) any later version.
%% The latest version of this license is in
%%   http://www.latex-project.org/lppl.txt
%% and version 1.3 or later is part of all distributions of LaTeX
%% version 2005/12/01 or later.
%%
%% This work has the LPPL maintenance status `maintained'.
%% 
%% The Current Maintainer of this work is the abnTeX2 team, led
%% by Lauro César Araujo. Further information are available on 
%% http://abntex2.googlecode.com/
%%
%% This work consists of the files abntex2-modelo-projeto-pesquisa.tex
%% and abntex2-modelo-references.bib
%%

% ------------------------------------------------------------------------
% ------------------------------------------------------------------------
% abnTeX2: Modelo de Projeto de pesquisa em conformidade com 
% ABNT NBR 15287:2011 Informação e documentação - Projeto de pesquisa -
% Apresentação 
% ------------------------------------------------------------------------ 
% ------------------------------------------------------------------------

\documentclass[
	% -- opções da classe memoir --
	12pt,				% tamanho da fonte
	openany,			%openright capítulos começam em pág ímpar (insere página vazia caso preciso)
	oneside,			%twoside para impressão em verso e anverso. Oposto a oneside
	a4paper,			% tamanho do papel. 
	% -- opções da classe abntex2 --
	%chapter=TITLE,		% títulos de capítulos convertidos em letras maiúsculas
	%section=TITLE,		% títulos de seções convertidos em letras maiúsculas
	%subsection=TITLE,	% títulos de subseções convertidos em letras maiúsculas
	%subsubsection=TITLE,% títulos de subsubseções convertidos em letras maiúsculas
	% -- opções do pacote babel --
	english,			% idioma adicional para hifenização
	french,				% idioma adicional para hifenização
	spanish,			% idioma adicional para hifenização
	brazil,				% o último idioma é o principal do documento
	]{abntex2}

% ---
% PACOTES
% ---

% ---
% Pacotes fundamentais 
% ---
\usepackage{lmodern}			% Usa a fonte Latin Modern
\usepackage[T1]{fontenc}		% Selecao de codigos de fonte.
\usepackage[utf8]{inputenc}		% Codificacao do documento (conversão automática dos acentos)
\usepackage{indentfirst}		% Indenta o primeiro parágrafo de cada seção.
\usepackage{color}				% Controle das cores
\usepackage{graphicx}			% Inclusão de gráficos
\usepackage{microtype} 			% para melhorias de justificação
\usepackage{tikz}
\usepackage{float}
\usetikzlibrary{shapes,arrows}
% ---

% ---
% Pacotes adicionais, usados apenas no âmbito do Modelo Canônico do abnteX2
% ---
\usepackage{lipsum}				% para geração de dummy text
% ---

% ---
% Pacotes de citações
% ---
\usepackage[brazilian,hyperpageref]{backref}	 % Paginas com as citações na bibl
\usepackage[alf]{abntex2cite}	% Citações padrão ABNT

% --- 
% CONFIGURAÇÕES DE PACOTES
% --- 

% ---
% Configurações do pacote backref
% Usado sem a opção hyperpageref de backref
\renewcommand{\backrefpagesname}{Citado na(s) página(s):~}
% Texto padrão antes do número das páginas
\renewcommand{\backref}{}
% Define os textos da citação
\renewcommand*{\backrefalt}[4]{
	\ifcase #1 %
		Nenhuma citação no texto.%
	\or
		Citado na página #2.%
	\else
		Citado #1 vezes nas páginas #2.%
	\fi}%
% ---

% ---
% Informações de dados para CAPA e FOLHA DE ROSTO
% ---
\titulo{Um estudo de caso do uso de mineração de dados e aprendizado de máquina
  no aprimoramento de  inspeções de estações radio base} \autor{Marcelo Veloso
  Maciel} \local{Brasil}

\instituicao{%
  Universidade de Pernambuco -- UPE
  \par
  Residência Tecnológica em Inteligência Artificial}
\tipotrabalho{Trabalho de conclusão}
% O preambulo deve conter o tipo do trabalho, o objetivo, 
% o nome da instituição e a área de concentração 
\preambulo{Trabalho de conclusão}
% ---

% ---
% Configurações de aparência do PDF final

% alterando o aspecto da cor azul
\definecolor{blue}{RGB}{41,5,195}

% informações do PDF
\makeatletter
\hypersetup{
     	%pagebackref=true,
		pdftitle={\@title}, 
		pdfauthor={\@author},
    	pdfsubject={\imprimirpreambulo},
	    pdfcreator={LaTeX with abnTeX2},
		pdfkeywords={abnt}{latex}{abntex}{abntex2}{projeto de pesquisa}, 
		colorlinks=true,       		% false: boxed links; true: colored links
    	linkcolor=blue,          	% color of internal links
    	citecolor=blue,        		% color of links to bibliography
    	filecolor=magenta,      		% color of file links
		urlcolor=blue,
		bookmarksdepth=4
}
\makeatother
% --- 

% --- 
% Espaçamentos entre linhas e parágrafos 
% --- 

% O tamanho do parágrafo é dado por:
\setlength{\parindent}{1.3cm}

% Controle do espaçamento entre um parágrafo e outro:
\setlength{\parskip}{0.2cm}  % tente também \onelineskip

% ---
% compila o indice
% ---
\makeindex
% ---

% ----
% Início do documento
% ----
\begin{document}

% Retira espaço extra obsoleto entre as frases.
\frenchspacing 

% ----------------------------------------------------------
% ELEMENTOS PRÉ-TEXTUAIS
% ----------------------------------------------------------
% \pretextual

% ---
% Capa
% ---
\imprimircapa
% ---

% ---
% Folha de rosto
% ---
\imprimirfolhaderosto
% ---

% ---
% NOTA DA ABNT NBR 15287:2011, p. 4:
%  ``Se exigido pela entidade, apresentar os dados curriculares do autor em
%     folha ou página distinta após a folha de rosto.''
% ---

% ---
% inserir lista de ilustrações
% ---
\pdfbookmark[0]{\listfigurename}{lof}
\listoffigures*
\cleardoublepage
% ---

% ---
% inserir lista de tabelas
% ---
\pdfbookmark[0]{\listtablename}{lot}
\listoftables*
\cleardoublepage
% ---

% ---
% inserir lista de abreviaturas e siglas
% ---
% ---

% ---
% inserir o sumario
% ---
\pdfbookmark[0]{\contentsname}{toc}
\tableofcontents*
\cleardoublepage
% ---


% ----------------------------------------------------------
% ELEMENTOS TEXTUAIS
% ----------------------------------------------------------
\textual

% ----------------------------------------------------------
% Introdução
% ----------------------------------------------------------
\chapter*[Introdução]{Introdução}
\addcontentsline{toc}{chapter}{Introdução}

%\backrefsetup{disable}

Nas últimas décadas a temática do impacto social da inteligência artificial vem
tomando centralidade no imaginário prospectivo do cidadão médio, da comunidade
científica e dos agentes estatais \cite{cameron1991terminator,
  cockburn2018impact, makridakis2017forthcoming}. A ascensão do assunto na
opinião pública não é desconexa de mudanças no contexto econômico e político
\cite{kogut2003global}. A difusão da internet na sociedade, culminando nas
tecnologias IoT \cite{gubbi2013internet}, faz com que dados passem a ser
consideradas pela The Economist \footnote{Fonte:
  \url{https://tinyurl.com/y39u52kk}.
  Acessado em 1 de Novembro de 2019 .} o novo petróleo.

Esse papel dos dados pressupõe a capacidade dos agentes econômicos de extrair
valor deles. É essa a seara de inserção dos algoritmos de inteligência
computacional, particularmente os de aprendizado de máquina. Algoritmos de
aprendizado de máquina são aqueles que aprendem com uma experiência com relação
a alguma tarefa e uma medida de performance se a performance na tarefa melhora
com a experiência \cite{carbonell1984machine}. Se os dados são o novo petróleo
então os algoritmos utilizados para extrair informação e aprender com esses
dados podem ser considerados os novos motores da economia.

Embora grandes empresas de tecnologia como Google, Facebook e Amazon façam uso
de grandes arquiteturas de redes neurais artificiais as quais necessitam de
dezenas de horas de treinamento em unidades de processamento gráfico, a
realidade da maior parte das empresas que buscam se inserir nessa nova era
algorítmica difere em escopo \cite{canziani2016analysis}. Se por um lado a
inteligência artificial traz a possibilidade de uma riqueza de aplicações e
otimizações no processo produtivo das empresas, por outro lado se faz necessária
uma infraestrutura de dados que permita a aplicação dessas técnicas e uma
``pipeline'' de mineração e recuperação de informação
\cite{schutze2007introduction}. Ademais a restrição orçamentária e computacional
e o imperativo da interpretabilidade\footnote{No contexto de aprendizado de
  máquina a interpretabilidade é definida por \citeonline[p.2]{doshi2017towards}
  "como a habilidade de explicar ou apresentar em termos compreensíveis para
  humanos". Uma definição equivalente de interpretabilidade é: o grau no qual um
  humano pode compreender a causa de uma decisão \cite{miller2018explanation}.}
do funcionamento dos algoritmos nos direciona, nesses casos medianos, à
algoritmos mais bem estabelecidos e simples em comparação aos de alta
publicização \cite{dreiseitl2002logistic}.

O presente estudo apresenta um caso de sucesso da aplicação de sistemas
inteligentes de recuperação e análise de informação de relativa simplicidade no
aprimoramento de um processo rotineiro na indústria de telecomunicações: a
inspeção de estações rádio base.

\chapter[Descrição do Caso]{Descrição do Caso}
Como referenciado anteriormente o sistema alvo de interesse do nosso estudo está
inserido no âmbito da indústria de telecomunicações. Na rede de celulares a
mediação entre o celular dos usuários e as companhias telefônicas é feita pelas
Estações Rádio Base (doravante ERB ou sítio celular). São nesses sítios que
estão instalados os equipamentos necessários para a comunicação entre aparelhos
celulares e as centrais de comunicação das agências telefônicas. Nesses
ambientes são realizadas vistorias frequentes tendo em vista sua relevância para
a qualidade do serviço de telefonia. Nessas vistorias são checados itens
referentes às chaves do sítio, à rua de acesso, alarmes externos, aterramento,
baterias, cabos, fontes de energia, antenas, dentre centenas outros. Essa
vistoria é um trabalho conjunto entre técnicos que visitam os sítios e
engenheiros de telecomunicação que analisam as informações. Atualmente essa
troca de informação é feita da seguinte maneira: o técnico visita a ERB e para
cada item de um \textit{checklist}, que tem de 400 a mais de 600 itens a
depender da empresa de telefonia detentora do sítio, tiram fotos que são
enviadas a um sistema, onde são aceitas ou rejeitadas pelos engenheiros na
central. Contudo, nem todo item precisa ser checado a depender de condições
particulares da ERB. Estes itens são, portanto, abonados.

Em conversas com técnicos e engenheiros responsáveis pelas inspeções foram
identificadas ao menos duas possibilidades de aplicação de inteligência
computacional no aperfeiçoamento do processo: a definição de quais itens são
abonados e quais são aprovadas ou rejeitadas. O problema da dispensa do item, enfoque do presente trabalho, é que os técnicos não sabem de antemão quais itens devem ser abonados em um determinado sítio. Ao chegarem a ERB, desta forma, primeiro devem checar dentre centenas de itens em uma lista quais são dispensáveis e só então iniciam o trabalho da vistoria propriamente dita. Isso contribui drasticamente para a lentidão da atividade. Nossa contribuição para a redução do tempo despendido nessa checagem é descrita em seguida.

\chapter[Solução proposta]{Solução proposta}
Temos por problema a determinação de quais itens de um checklist são passíveis
de abono. Isso pode ser modelado como um problema de classificação binária :
dado um conjunto de características de um sítio e qual o item desejamos prever
se ele é da classe ``abonado'' ou não \cite{james2013introduction}. Especialistas apontaram a seguinte lista de características de um sítio que os próprios técnicos usam para abonar manualmente os itens:

\begin{itemize}
\item Tipo de site: Gabinete ou Container;
\item Tipo de tecnologia: WCDMA, LTE, GSM;
\item Frequência: 450Mhz, 700Mhz, 850Mhz, 1800Mhz,  2100Mhz, 2600Mhz;
  \item Equipamentos: Diplex, Triplex, Quadriplex, EHCU, Filtro, TMA, DTMA
\end{itemize}

Essas informações, contudo, não estão prontamente disponíveis. Uma fonte
possível de informação são os Projetos Preliminares de Instalação (PPIs). Eles
estão disponíveis em um sistema interno das empresas de telefonia, ao qual nos
foi dado acesso, em formato pdf. Tivemos acesso também à base de checklists dos
sítios. Identificamos 602 ERBs cadastrada nesse sistema das quais baixamos cerca
de 150 checklists e PPIs. Dentre os PPIs foram identificados 3 padrões de
documento. Como um esforço inicial trabalhamos na extração de informação de um
único padrão. Dado esse recorte de um único tipo de documento, a intersecção
entre o grupo de sítios que tínhamos tanto o checklist quanto o PPI tem uma
cardinalidade de 44, o tamanho da nossa amostra.



\chapter*[Conclusão]{Conclusão}
\addcontentsline{toc}{chapter}{Conclusão}

% ---
% Conclusão
% ---

% ELEMENTOS PÓS-TEXTUAIS
% ----------------------------------------------------------
\postextual

% ----------------------------------------------------------
% Referências bibliográficas
% ----------------------------------------------------------
\bibliography{abntex2-modelo-references}

% ----------------------------------------------------------
% Glossário
% ----------------------------------------------------------
%
% Consulte o manual da classe abntex2 para orientações sobre o glossário.
%
%\glossary
%---------------------------------------------------------------------
% INDICE REMISSIVO
%---------------------------------------------------------------------

\phantompart

\printindex


\end{document}
